\documentclass[reqno]{amsart}

\usepackage[a4paper]{geometry}

%\usepackage{xr-hyper} %load before hyperref
\usepackage[hidelinks,pdfpagelabels]{hyperref} %hyphenation rules british english

%%Language, Encoding, and Text Formatting
\usepackage[british]{babel} %hyphenation rules for british english
%\usepackage[spanish,mexico]{babel} %hyphenation rules mexican spanish
\usepackage[latin1]{inputenc} %.tex file encoding (input)
\usepackage[T1]{fontenc} %.pdf/.dvi file encoding (output)
%\usepackage{datetime} %date and time formatting

%%Mathematics
\usepackage{amsmath,amssymb,amsfonts,amsthm} %All AMS packages
\usepackage{mathrsfs} %to use cursive s
%\usepackage{bbm} %to use the bold 1

%Graphics
%\usepackage[all,2cell,cmtip]{xy} %xy-pic (place after babel)
%\UseAllTwocells
%\UseCrayolaColors
%\CompileMatrices
%\usepackage{tikz}

%%Document Format
\allowdisplaybreaks %lets \[ \] be broken into different pages
%\usepackage[svgnames]{xcolor} %Colors with the svgnames palette
%\numberwithin{equation}{section}
\renewcommand{\theenumi}{\roman{enumi}}
\usepackage{enumitem}

%%Theorems
%\theoremstyle{plain}
%\newtheorem{teo}{Theorem}[section]
%\newtheorem{prop}[teo]{Proposition}
%\newtheorem{lem}[teo]{Lemma}
%\newtheorem{cor}[teo]{Corollary}
%\theoremstyle{remark}
%\newtheorem{eje}[teo]{Example}
%\newtheorem{obs}[teo]{Observation}
%\newtheorem{rem}[teo]{Remark}
%\newtheorem*{note}{Note}
%\theoremstyle{definition}
%\newtheorem{defi}[teo]{Definition}
%\newtheorem*{notat}{Notation}

%%Proofs
%\renewcommand{\qedsymbol}{$\blacksquare$}
%\renewcommand{\qed}{\hfill\qedsymbol\newline}
%\renewcommand{\qedhere}{\hfill\qedsymbol}
\renewenvironment{proof}{\textit{Proof.}$\;$}{\qed}

%%Abbreviations

%%Shorthands

%%Metadata
\author{Ram�n Abud Alcal�}
\title{Q-CTRL\\ Challenge Summary}
\date{\today}
%\date{\formatdate{03}{12}{2017}} %This is how to get a specific date \usepackage{datetime}
%\thanks{
%\textcolor{red}{Easy, write later.}
%\textcolor{blue}{Nasty, consider removal.}
%\textcolor{violet}{Ugly, rewrite or reword.}
%\textcolor{teal}{Unclear, ask or think later.}
%}

%Playspace


\begin{document}
\maketitle

This is my attempt on the Q-CTRL challenge, instructions are found here \url{https://docs.q-ctrl.com/boulder-opal/application-notes/q-ctrl-qchack-challenge}.

Some of the resources that helped me understand what is really happening are:
\begin{itemize}
\item \url{https://en.wikipedia.org/wiki/Qubit#Physical_implementations}
\item \url{https://en.wikipedia.org/wiki/Electromagnetic_pulse#Non-nuclear_electromagnetic_pulse_(NNEMP)}
\item Pulse optimization for error-robust control on cloud-based hardware. \url{https://www.youtube.com/watch?v=ZemwL5mKNXM}
\end{itemize}

Before this I was thinking of qubits as being unitary vectors in a 2-dimensional complex space, that is, linear combinations of $\vert 0 \rangle$ and $\vert 1 \rangle$ with complex coefficients with norm 1. Hence, I was thinking of qubits as points in a 4-dimensional sphere. So, most of the time I had to rely on my intuition from only having real coefficients and a 2-dimensional sphere. But what I didn't know is that we can remove another degree of freedom by getting rid of the global phase. So qbits can be thought of a slice of the 4-sphere: a 3-sphere. The standard representation of qbits in this way is called a Poincare-Bloch sphere.

% https://en.wikipedia.org/wiki/Hopf_fibration

Unfortunately I did not have enough time to understand all what was needed to finish the challenge. I did manage to get a second pulse for the Hadamard gate which I call \(H2\). I followed the jupyter notes and all changes are made there.
It seems that \(H2\) has a standard error marginally better than that of the \(H\) provided.\\
\texttt{
H estimated probability of getting 1 is 0.4609375\\
H estimate standard error: 0.015577243301708272\\
H2 estimated probability of getting 1 is 0.490234375\\
H2 estimate standard error: 0.015622019483489914\\
}

\end{document}